\documentclass[a4paper]{article}
\usepackage[utf8]{inputenc}
\title{}
\author{Philipp Trommler}
\date{\today}


\begin{document}
\section{Problemstellung verstehen (1.-2. Woche)}
\subsection{Literatur lesen (1. Woche)}
\subsubsection{Bachelor-Arbeit}
\subsubsection{X-ray paper}
\subsubsection{FT}
\subsubsection{Gauss'sche Kernel}
\subsubsection{CUDA streams/async memory}
\subsubsection{cuFFT/cuBLAS}
\subsection{Programmaufbau festlegen (2. Woche)}
\subsubsection{SW in Abschnitte teilen}
\subsubsection{Parallelisierungspotenziale erkennen}
\paragraph{Datenabhängigkeiten erkennen}
\paragraph{Problemaufteilung maximieren}
\subsubsection{memory access koordinieren}
\subsubsection{Speicherformat festlegen}
binäres Pixmap (Versuchsdatenformat beachten)
\section{Implementierung (3.-7. Woche)}
\subsection{Beispiel-Daten erstellen (3. Woche)}
\subsubsection{Beispiele festlegen}
\paragraph{symmetrische}
Kugel,
Kegel
\subparagraph{periodische}
Schachbrett
\paragraph{asymmetrische}
halbe Kugel,
halber Kegel,
grobes Rauschen
\subsubsection{python-Programm schreiben}
\subsection{Algo-Implementierung (4.-7. Woche)}
\subsubsection{IO}
\subsubsection{HIO-Implementierung}
\subsubsection{Shrink-Wrap}
\subsubsection{Verifizierung der Ergebnisse anhand der Beispieldaten}
\subsubsection{Benchmarking}
\subsubsection{falls Zeit reicht: Performance-Analye und -Optimierung}
\subsection{Doku schreiben (parallel)}
\subsection{Examples schreiben (parallel)}
\section{Abschlusspräsentation (8. Woche)}
\subsection{Bilder}
\subsubsection{Benchmarks durchführen}
\paragraph{Referenzdaten (z.B. aus Bachelorarbeit) sichten und vergleichen}
\subsubsection{Ausgabedatenvisualisierung}
\end{document}
