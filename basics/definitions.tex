

\usepackage[utf8]{inputenc}
\usepackage{uniinput}
\usepackage{geometry}
%\usepackage{fancyhdr}
%\usepackage{blindtext}
%\usepackage{setspace}
\usepackage{cmap} % pro­vides char­ac­ter map ta­bles, which makes PDFs search­able and copy-able (I don't see a difference yet!!!)
\setlength\parindent{0pt} % stop indenting every paragraph by 1cm. It looks awful!

\usepackage{graphicx}
\usepackage{tikz}
\usetikzlibrary{arrows, decorations.markings, calc, datavisualization, datavisualization.formats.functions}
%\usepackage{pgfplots}   %plotting with tex

\usepackage{amsmath}
\usepackage{amsfonts}
\usepackage{amssymb}
%\usepackage{esint}
\usepackage{ulem}
\usepackage{bbm}		% sames as above, but won't redefine the better looking ams-mathbb, doesn't contain \mathbbm{0} though
\usepackage{cancel}		% for making a stroke through a calculation
%\usepackage{tensor}
\allowdisplaybreaks     % allow page breaks for longer align-environments

\usepackage{tabularx}
\newcolumntype{Y}{>{\centering\arraybackslash}X} % centered column of adjustable size i.e. with auto line break
\usepackage{multirow}
\usepackage{multicol}

\usepackage{xcolor}
\usepackage{rotating}	% rotate imported picture
\usepackage{pdflscape}	% rotate site by 90° with \begin{landscape}
\usepackage{fancyhdr}
\usepackage[binary-units=true]{siunitx}    % SI units paper conform with \SI{3}{\nano\seconds} 
                                           % sudo apt-get install texlive-science
\usepackage[labelfont=bf,format=plain]{caption} % enables \captionof for non figure/table environments
%\numberwithin{equation}{chapter}    % equation number contains chapter number
%\numberwithin{figure}{chapter}
%\numberwithin{table}{chapter}

%\usepackage{textcomp}	%for \textdegree
%\usepackage{gensymb}	%for \degree in math mode

\usepackage{url}		%for \url{} command to not get problems with :\\ and other symbols
%\usepackage{breakurl}  % only needed when compiling via dvips/ps2pdf
\usepackage{hyperref}	%for labellinks

\usepackage{attachfile} % insert attachments into pdf O_O
\usepackage{animate}



%%%%%%%%%%%%%%%%%%%%%%%%%%%%%%%%%%%%%%%%%%%%%%%%%%%%%%%%%%%%%%%%%%%%%%%%%%%%%%%%


%\renewcommand{\vec}[1]{\overrightarrow{#1}}	%overrightarrow is basically just a bigger vector arrow
\newcommand{\important}[1]{\textcolor{BrickRed}{#1}}
\newcommand{\mat}[1]{\mathbf{#1}}

%\renewcommand{\i}{\text{i}} %normal math i is without point

\newcommand{\de}{\text{d}}
\newcommand{\const}{\text{\small const.}}
\renewcommand{\div}{\text{div}\,}
\renewcommand{\nabla}{\overrightarrow{\bigtriangledown}}
\newcommand{\laplace}{\bigtriangleup}
\newcommand{\grad}{\text{grad}\,}
\newcommand{\rot}{\text{rot}\,}
\newcommand{\dalembert}{\square}
\newcommand{\affects}[1]{\overset{\downarrow}{#1}}
\newcommand{\vecb}[2]{\left(\vec{#1}\right)_{#2}}
\newcommand{\fpart}[2]{\frac{\partial #1}{\partial #2}}
\newcommand{\ftdif}[2]{\frac{\de #1}{\de #2}}
\newcommand{\slfrac}[2]{\left.#1\middle/#2\right.}
\newcommand{\scos}[1]{\text{c}_{#1}\,}
\newcommand{\ssin}[1]{\text{s}_{#1}\,}

%ToDo: Switch between shorter symbols for inverse functions
\newcommand{\invf}[1]{#1^{-1}}				%Inverse function notation

\newcommand{\arsinh}{\text{arsinh}\,}
\newcommand{\arcosh}{\text{arcosh}\,}
\newcommand{\arcoth}{\text{arcoth}\,}
\newcommand{\artanh}{\text{artanh}\,}

\newcommand{\atan}{\text{atan}\,} %short for ArcTan
\newcommand{\asin}{\text{asin}\,}
\newcommand{\acos}{\text{acos}\,}

%Physik
\newcommand{\angstrom}{\overset{\circ}{\text{A}}}
\newcommand{\tesla}{\text{T}}
%\newcommand{\lagrangian}{\mathcal{L}}
%\newcommand{\lagsecondkind}{\ftdif{}{t}\fpart{\lagrangian}{\dot{q_j}}-\fpart{\lagrangian}{q_j}}


\newcommand{\R}{\mathbb{R}}
\newcommand{\smd}[2]{\sum\limits_{#1}^{#2}} %SuM with Defined boundaries
\newcommand{\ind}[2]{\int\limits_{#1}^{#2}} %INtegral with Defined boundaries

\newcommand{\ex}{\hat{x}}
\newcommand{\ey}{\hat{y}}
\newcommand{\ez}{\hat{z}}

%For use in matrices to denote vectors inside a matrix
\newcommand*{\vertbar}{\vrule}%\rule[1ex]{0.5pt}{2.5ex}}
\newcommand*{\horzbar}{\rule[.5ex]{2.0ex}{0.5pt}}	%[Vertical-Shift]{Length}{Thickness}

%Unicodes
\DeclareUnicodeCharacter{2208}{\ensuremath{\in}}
\DeclareUnicodeCharacter{2115}{\ensuremath{\mathbb N}}
\DeclareUnicodeCharacter{2124}{\ensuremath{\mathbb Z}}
\DeclareUnicodeCharacter{2192}{\ensuremath{\rightarrow}}

%QT
\newcommand{\mean}[1]{\langle #1 \rangle}
\newcommand{\esk}[1]{\textcolor{BrickRed}{#1}}	%Einsteinsche Summenkonvention Marker
\usepackage{braket} %Bra,Ket,Braket(scaled) vs. bra,ket,braket(smaller)
\usepackage{MnSymbol} %sumint

%Math
\newcommand{\HR}{\mathcal{H}} %Hilbertraum
\newcommand{\solvefor}[1]{\textcolor{Blue}{#1}}



%%%%%%%%%%%%%%%%%%%%%%%%%%%%%%%%%%%%%%%%%%%%%%%%%%%%%%%%%%%%%%%%%%%%%%%%%%%%%%%%


\definecolor{mygreen}{rgb}{0,0.6,0}
\definecolor{mygray}{rgb}{0.5,0.5,0.5}
\definecolor{mymauve}{rgb}{0.58,0,0.82}

\usepackage{listings}   % source code listings
\usepackage{textcomp}   % for upquotes inside lstlisting

\usepackage{courier}    % for lstlisting

\lstset{ %
  backgroundcolor=\color{white},   % choose the background color; you must add \usepackage{color} or \usepackage{xcolor}
  basicstyle=\footnotesize\ttfamily, % the size of the fonts that are used for the code
  breakatwhitespace=false,         % sets if automatic breaks should only happen at whitespace
  breaklines=true,                 % sets automatic line breaking
  captionpos=b,                    % sets the caption-position to bottom
  commentstyle=\color{mygreen},    % comment style
%  deletekeywords={...},            % if you want to delete keywords from the given language
  escapeinside={\%*}{*)},          % if you want to add LaTeX within your code
  extendedchars=false,              % lets you use non-ASCII characters; for 8-bits encodings only, does not work with UTF-8
  %frame=single,                    % adds a frame around the code
  keepspaces=true,                 % keeps spaces in text, useful for keeping indentation of code (possibly needs columns=flexible)
  keywordstyle=\color{blue},       % keyword style
  morekeywords={*,...},            % if you want to add more keywords to the set
  numbers=left,                    % where to put the line-numbers; possible values are (none, left, right)
  numbersep=5pt,                   % how far the line-numbers are from the code
  numberstyle=\tiny\color{mygray}, % the style that is used for the line-numbers
  rulecolor=\color{black},         % if not set, the frame-color may be changed on line-breaks within not-black text (e.g. comments (green here))
  showspaces=false,                % show spaces everywhere adding particular underscores; it overrides 'showstringspaces'
  showstringspaces=false,          % underline spaces within strings only
  showtabs=false,                  % show tabs within strings adding particular underscores
  stepnumber=1,                    % the step between two line-numbers. If it's 1, each line will be numbered
  stringstyle=\color{mymauve},     % string literal style
  tabsize=2,                       % sets default tabsize to 2 spaces
  title=\lstname,                  % show the filename of files included with \lstinputlisting; also try caption instead of title
  xleftmargin=.25in,               % indent listings by default
  upquote=true                     % single quotes '' appear straight instead of as backticks. Needs textcomp
}

